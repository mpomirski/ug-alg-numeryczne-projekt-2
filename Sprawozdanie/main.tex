% Preamble
\documentclass[titlepage]{article}
\usepackage[a4paper, total={6in, 8in}]{geometry}
\usepackage{graphicx, polski, inputenc, hyperref, titlepic, listings, amsmath, amsfonts, svg, blkarray, fancyhdr, enumitem}

% Page header and footer
\pagestyle{fancy}
\fancyhf{}
\rhead{Kwiecień 2024}
\lhead{\bfseries Algorytmy numeryczne - Niezdecydowany wędrowiec}
\fancyfoot{}
\renewcommand{\footrulewidth}{0.4pt}
\lfoot{Konrad Kreczko, Michał Pomirski}
\rfoot{}

% Listings settings
\lstset{
    language=go,
    basicstyle=\footnotesize,
    numbers=left,
    stepnumber=1,
    showstringspaces=false,
    tabsize=1,
    breaklines=true,
    breakatwhitespace=false,
}

% Document content
\begin{document}
% -------------- Title page --------------
\thispagestyle{empty}
\begin{center}
    {\sc \Large UNIWERSYTET GDAŃSKI}
    \par\vspace{1cm}\par
    {\large
    WYDZIAŁ MATEMATYKI, FIZYKI I INFORMATYKI\\
    Kierunek: Informatyka Praktyczna\\
    Rok II Semestr 4\\
    Algorytmy Numeryczne
    }
\end{center}
\begin{center}
    \includegraphics[width=5cm]{ug-logo.png}
\end{center}
\begin{center}
    {\large
    Konrad Kreczko, Michał Pomirski
    }\par\vspace{0.5cm}\par

    {\LARGE
    NIEZDECYDOWANY WĘDROWIEC
    }
\end{center}
\vspace{4cm}
\vfill
\begin{center}
    Gdańsk 2024
\end{center}

\newpage
% -------------- Table of contents --------------
\tableofcontents
\newpage

\section{Opis problemu}
\subsection{Wersja uproszczona}
Niezdecydowany wędrowiec spacerując parkową aleją położoną w linii NS porusza się jeden krok na północ z prawdopodobieństwem 1/2 i z prawdopodobieństwem 1/2 jeden krok na południe. Przyjmujemy, że wędrowiec robi równe kroki w obu kierunkach, a kolejne losowania są niezależne.
\par W odległości $n$ kroków na północ od początkowego położenia (start) wędrowca znajduje się odkryta studzienka kanalizacyjna (OSK) do której niezdecydowany wędrowiec z pewnością wpadnie, złamie nogę i tak skończy się ten spacer, jeśli tylko do niej dotrze.
\par W odległości $s$ kroków na południe od początkowego położenia wędrowca znajduje się wyjści z parku. Po dojściu do wyjścia wędrowiec przestaje być niezdecydowany i dalej pewnym krokiem bezpiecznie wraca do domu.
\par Interesuje nas prawdopodobieństwo, że wędrowiec bezpiecznie wróci do domu.


\subsection{Wersja podstawowa}
Park składa się z $m$ alejek, które krzyżują się w punktach $v_1, v_2, ..., v_n$. Początek i koniec alejki uważamy za skrzyżowanie. Jest co najwyżej jedna alejka łącząca $v_i$ z $v_j$ o długości $d_{ij}$ kroków, $d_{ij} \in \mathbb{N}^+$.
\par Niezdecydowany wędrowiec spacerując parkową aleją porusza się podobnie jak w wersji uproszczonej z prawdopodobieństwem 1/2 w jedną lub drugą stronę. Jeśli trafi na skrzyżowanie, to wybiera z równym prawdopodobieństwem jedną z alejek (łącznie z tą z której trafił na skrzyżowanie).\\
Jedno ze skrzyżowań jest punktem startowym, jedno ze skrzyżowań jest wyjściem.\\ W jednym ze skrzyżowań znajduje się OSK (ale nie w wyjściu i nie w punkcie startowym).\\
Podobnie jak w wersji uproszczonej, interesuje nas prawdopodobieństwo, że wędrowiec bezpiecznie wróci do domu.\\
Dodatkowo, możliwe są poniższe rozszerzenia danego problemu:
\begin{itemize}
\item[$R0$:]Może się zdarzyć, że w parku jest więcej niż jedno wyjście.
\item[$R1$:]Może się zdarzyć, że w parku jest więcej niż jedna OSK.
\item[$R2$:]W parku może być dwóch wędrowców, należy wyznaczyć prawdopobieństwo, że oboje wrócą do domu. W trakcie wędrówki, jeśli oboje miną się w tej samej alejce, staną w tym samym miejscu alejki lub na tym samym skrzyżowaniu, dochodzi do awantury i incydentu, który nie kończy się bezpiecznym powrotem do domu.
\item[$R3$:]Może się zdarzyć, że kilka alejek, być może różnej długości, łączy te same dwa skrzyżowania.
\item[$R4$:]Przy wybranych skrzyżowaniach stoją pojemniki na śmieci. Wędrowiec nie lubi
śmietników i prawdopodobieństwo, że zrobi krok do miejsca, gdzie pojemnik się znajduje jest dwa razy mniejsze niż w stronę, gdzie pojemnika nie ma. Należy uwzględnić przypadek bardzo krótkich alejek, np. o długości jednego lub dwóch kroków.
\end{itemize}

\newpage
\section{Podejście do problemu}
\subsection{Wersja uproszczona}
Dla $n = 2$ i $s = 3$, oznaczymy miejsca w których może stać wędrowiec jako: $m_0\ (wyjście)$, $m_1$, $m_2$, $m_3\ (start)$, $m_4$ i $m_5\ (OSK)$.\\
Niech $p_i$ oznacza prawdopodbieństwo bezpieczenego powrotu do domu, gdy wędrowiec stoi w punkcie $m_i$.\\
Możemy zapisać następujące równania:
\begin{align}
&p_0 = 1,\\
&p_5 = 0,\\
&p_1 = 0.5\cdot p_0 + 0.5 \cdot p_2
\end{align}
W postaci macierzowej, układ równań odzwierciedlający podaną sytuację wygląda tak:
\begin{align}\label{eq4}
\begin{bmatrix}
1 & 0 & 0 & 0 & 0 & 0 \\
-0.5 & 1 & -0.5 & 0 & 0 & 0 \\
0 & -0.5 & 1 & -0.5 & 0 & 0 \\
0 & 0 & -0.5 & 1 & -0.5 & 0 \\
0 & 0 & 0 & -0.5 & 1 & -0.5 \\
0 & 0 & 0 & 0 & 0 & 1 
\end{bmatrix}
\begin{bmatrix}
p_0\\
p_1\\
p_2\\
p_3\\
p_4\\
p_5
\end{bmatrix}
=
\begin{bmatrix}
1\\
0\\
0\\
0\\
0\\
0
\end{bmatrix}
\end{align}
Rozwiązaniem tego układu jest wektor prawdopodobieństw $P$, który dla przypadku powyżej wynosi:
\begin{align}
P=
\begin{bmatrix}
1\\
0.8\\
0.6\\
0.4\\
0.2\\
0
\end{bmatrix}
\end{align}
Zatem, dla danego punktu startowego $s=3$, prawdopobieństwo bezpieczenego powrotu do domu wynosi $0.4$.
\newpage
\subsection{Błądzenie losowe w grafach}
Opisany problem wędrowca jest przykładem problemu błędzenia losowego w grafach, związanego z teorią obwodów elektrycznych, czy algorytmu PageRank.
\par Problem ten można przedstawić w postaci grafu skierowanego, gdzie skrzyżowania przedstawione są jako wierzchołki, a alejki - jako krawędzie z wagami odpowiadającymi prawdopodobieństwu przejścia między skrzyżowaniami.
\begin{figure}[h]
    \centering
    \includesvg[width=5.0in]{graph.svg}
    \caption{Przedstawienie wersji uproszczonej problemu w postaci grafu skierowanego.}
    \label{fig:enter-label}
\end{figure}\\

Przedstawiając taki graf jako macierz sąsiedztwa, otrzymujemy tę samą macierz co w \eqref{eq4}.
\[
\begin{blockarray}{ccccccc}
 & $m_0$ & $m_1$ & $m_2$ & $m_3$ & $m_4$ & $m_5$ \\
\begin{block}{c[cccccc]}
$m_0$ & 1 & 0 & 0 & 0 & 0 & 0 \\
$m_1$ & -0.5 & 1 & -0.5 & 0 & 0 & 0 \\
$m_2$ & 0 & -0.5 & 1 & -0.5 & 0 & 0 \\
$m_3$ & 0 & 0 & -0.5 & 1 & -0.5 & 0 \\
$m_4$ & 0 & 0 & 0 & -0.5 & 1 & -0.5 \\
$m_5$ & 0 & 0 & 0 & 0 & 0 & 1 
\end{block}
\end{blockarray}
\]

\section{Algorytmy}

\subsection{Eliminacja Gaussa bez wyboru elementu podstawowego}
Eliminacja Gaussa bez wyboru elementu podstawowego

\subsection{Eliminacja Gaussa z częściowym wyborem elementu podstawowego}
Eliminacja Gaussa z częściowym wyborem elementu podstawowego

\subsection{Metoda iteracyjna Gaussa-Seidela}
Metoda iteracyjna Gaussa-Seidela

\newpage

\section{Hipotezy}

\subsection{Hipoteza 1}
Algorytm A2 zwykle daje dokładniejsze wyniki niż A1. Różnica w dokładności rośnie wraz z rozmiarem macierzy i liczbą niezerowych współczynników.
\\
\par Dowód fajny tak.

\subsection{Hipoteza 2}
Algorytm A3 zawsze działa dla podstawowego zadania.
\\
\par Dowód fajny tak.

\subsection{Hipoteza 3}
Jeśli Algorytm A3 jest zbieżny do rozwiązania, to wyniki otrzymujemy istotnie szybciej niż dla A1 i A2.
\\
\par Dowód fajny tak.

\section{Podsumowanie}
Podsumowanie

\section{Zakres pracy}
\begin{itemize}[label=\textbullet, leftmargin=1.2in]
    \item[Konrad Kreczko:] Implementacja algorytmu A1, Metoda Monte Carlo, Implementacja wersji podstawowej
    \item[Michał Pomirski:] Implementacja algorytmu A2, Metoda Monte Carlo, Sprawozdanie
\end{itemize}

\end{document}
